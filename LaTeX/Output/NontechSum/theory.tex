
  \section{Theoretical Framework}

  Perhaps the biggest financial decision in a household's life is buying a house. Houses come in different sizes (and therefore costs), but they are generally an expensive asset whose value is at least a few times the household's yearly income. Young households usually cannot buy their houses outright, as they start with little to no assets and take time to accumulate enough resources. They therefore usually rely on mortgages to purchase their houses. These young and leveraged households might thus be sensitive to stock market risk, causing them to reduce their stock market participation.

  During the repayment period, households' market resources (or liquid assets) are reduced by at least the fixed mortgage payment. This has two effects on risky asset choice: 1) It reduces 'market resources' (current income, plus current non-housing wealth) today, making households relatively less wealthy, and 2) It reduces market resources in following periods for any given level of current savings, making households more risk averse due to the precautionary motive. The fixed nature of mortgage payments has another important effect: since the household must be sure to have sufficient resources in the next period to pay their mortgage and the cost of house maintenance, they might want to save more of their resources in a safe account rather than in the risky stock market.

  Houses are also subject to price fluctuations that depend on local housing market conditions as well as broader national trends, such as recessions and expansions. The uncertain sale price of your house, then, constitutes additional uncertainty in future net worth and could thus have significant implications for portfolio decisions. One way to think about this is to realize that the owner of a house has an implicit holding of a risky asset, which should motivate them to want to reduce their exposure to additional risk in another risky asset, the stock market.


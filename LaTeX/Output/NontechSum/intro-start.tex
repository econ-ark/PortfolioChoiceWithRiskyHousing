  \section{Introduction}

  Economists have long sought to use models of mathematically optimal behavior to try to understand saving and investment decisions over a household's life-cycle\footnote{See \cite{Gomes2020a} for a survey of portfolio choice over the life-cycle.}. If there were no uncertainty (about, say, investment returns), calculating optimal choices would not be  hard.  For example, consumers would want to invest their whole financial portfolio in whatever asset they knew (in advance) would yield the highest rate of return.

  But in the real world, assets that -- on average -- yield higher returns (like stocks), also are much riskier than low-return safe assets (like bank deposits).  Aversion to risk is a perfectly rational motivation, so how much to invest in risky versus safe assets is far from obvious.  Furthermore, there are many other risks (to  job,  health, to house prices, and more) that should further temper any rational person's appetite for risky investment.

  As noted in \cite{Carroll2020}\footnote{This blog post replicates results from \cite{Cocco2005}.}, calculating truly optimal behavior in a realistically uncertain world is such a difficult challenge that only recently has it become feasible to do with a reasonably high degree of realism.  Rational choice models like the one we examined there must account for many important features of reality, including different types of uncertainty (labor income risk, mortality risk, and stock market risk), and should allow for reasonable choices of risk aversion, impatience, and other preferences.  They need properly to account for for the path of income over the life cycle and into retirement, effects of aging and mortality, interest rates and economic growth, and myriad other factors.

  All of this is so difficult that professional financial advisors do not attempt it, relying instead on rules of thumb and intuition to guide their clients.
  Indeed, despite the current excitement about the wonders of artificial intelligence, even  online ``robo-advisors'' do not incorporate the degree of realism described above.  Serious mathematical optimization efforts have been restricted to the pages of top academic economics journals -- and the associated computer code used to solve the models in those papers has been so impenetrable as to be unusable.


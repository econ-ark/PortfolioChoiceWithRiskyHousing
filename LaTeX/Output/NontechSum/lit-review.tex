
  \section{Literature Review}

  Beginning with \cite{Merton1969} and \cite{Samuelson1969}, there is an extensive literature on portfolio choice over the life-cycle. \cite{Cocco2005} develop a model of portfolio choice under incomplete markets and with labor income risk \footnote{Other life-cycle portfolio choice models are \cite{gomes2005optimal} and \cite{yao2005optimal}}. Although they are successful in solving a realistically calibrated life-cycle portfolio choice model, their results imply that most households, and in particular young households with low liquid wealth should invest \textit{all} of their assets in the stock market. In reality, however, people choose a degree of stock market participation and risky share of assets much lower than what the model proposes would be optimal even for a person with very high risk aversion\footnote{See \cite{Carroll2020}.}. This gap between observed and actual stock market participation is known as the stock market participation puzzle, and it remains an open question that is not explained by state-of-the-art quantitative life-cycle models.

  One possible explanation for the stock market participation puzzle could be the absence of additional forms of asset holding and risk exposure that households face. In particular, housing has dual properties: Both as an asset and as a source of consumption services . Housing, however, is different from other assets in that it is illiquid and durable, providing both future expected wealth (house value of liquidation) as well as shelter as a consumption service. \cite{Chetty2017} empirically quantify the effect of housing on portfolio choice. Looking at the Survey of Income and Program Participation (SIPP) panel from 1990 to 2008, the authors establish the importance of property value and home equity (value minus mortgage debt) on stock market participation. Importantly, they find that a \$10,000 increase in mortgage debt (holding home equity constant) causes the risky portfolio share to decrease by 0.6 percentage points or \$275, which amounts to 3.9\% of mean stockholdings in their data. Importantly for our purposes, they establish the need to distinguish the effects of home equity and mortgage debt in order to quantify the effect of housing on portfolios.

  This project builds a quantitative model of housing and portfolio choice that can be used to interpret the empirical findings \cite{Chetty2017} in a rich framework that includes liquid wealth, illiquid housing (size and value), and mortgage debt, in order to capture the effects of home equity and mortgage debt on portfolio choice. While other recent work has attempted to shed light on this topic, this model is the first of its kind to explicitly track home value and mortgage debt separately, allowing them each to evolve with the decisionmaker's choices and with economic shocks (say, to house prices). The 2-period model \cite{Chetty2017} develop misses the life-cycle properties of housing choice and income risk. Additionally, although their empirical findings reveal the importance of distinguishing home equity and mortgage debt, their model does not allow these variables to be chosen. \cite{Cocco2004rfs} and \cite{yao2005optimal} do allow choice of house size but do not distinguish between home equity and mortgage debt, instead only keeping track of net wealth as the total value of assets minus liabilities.


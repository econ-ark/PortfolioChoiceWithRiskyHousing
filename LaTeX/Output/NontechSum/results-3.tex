
  \subsection{House size crowds out investment}

  However, when comparing households on a total wealth basis, i.e.\ their liquid assets plus expected house liquidation, we can see that house size crowds out investment for households with low liquid wealth. In figure~\ref{fig:liquidAssetsByEquity}, we can consider a household whose house size is equal to 5 (5 times their yearly net income) and liquid assets are 0, so their total expected wealth is 5. As this household becomes wealthier, they invest all of their liquid assets in the stock market (such that their risky share is 100 percent), up to the point where they start rebalancing their portfolio between the risky and the safe asset. In this region, they are constrained from investing in the stock market by their low liquid wealth, as they surely would like to invest more in the market. This point becomes clearer by comparing the household to an equally wealthy peer with a smaller house.  At the point where the household with house size of 5 has liquid wealth of 1 (1 times their yearly net income), they are investing less into the stock market in absolute terms than an equally wealthy household whose house size is equal to 2 and liquid assets are equal to 4. The total expected wealth of both these households is 6, but the household with the larger house is investing fewer assets in the stock market than the household with the smaller house. As their total wealth increases, however, both households are unconstrained by their house size and end up investing about the same amount into the stock market in absolute terms.

  \providecommand{\figName}{}
  \renewcommand{\figName}{liquidAssetsByEquity}
  \providecommand{\figFile}{}
  \renewcommand{\figFile}{\figName}
  \input{\FigDir/\figName} % Read in the tex to generate the figure



  \section{Conclusions}

  Most people who need advice about how to invest in financial markets are also homeowners. But until now, even the most sophisticated and realistic analyses of how people should optimally invest in financial markets have not accounted for the (undeniably important) ramifications of homeownership for their financial choices.

  That's because constructing a model that correctly tracks all the potential interactions between homeownership, financial risk, and other kinds of risk is remarkably difficult. This report describes a free, publicly available open-source software tool that does these complex calculations.  Sponsorship by the Think Forward Initiative has allowed us to add this tool to the free, open-source, \href{https://econ-ark.org}{Econ-ARK} toolkit, thus making it available to financial institutions, financial planners, robo-advisors, academics, and anyone else who might be interested in a rigorous analysis of these questions.

  Despite their combinatorial complexity, the answers that come from the model make intuitive sense. A first conclusion is that greater uncertainty about future house prices should make you less willing to invest in the stock market.  In other words, a homeowner who lives in a place with wild house-price swings will find it best to have less exposure to other kinds of risk (like stock market risk) than someone with circumstances that are otherwise similar, but who lives in a place where house prices are more stable.

  Another conclusion might seem to push in the other direction, but really does not: Among homeowners whose mortgage is paid off, for a given level of \textit{nonhousing} net worth (say, \$200K of financial assets net of mortgage debt), a person whose house is more valuable should invest more in risky financial asset.  The reason is simple:  For a given amount of liquid assets, the person with a bigger house is richer, and a richer person will want to have more money (in absolute terms) invested in the stock market).

  The final point is that the existence of homeownership does not reverse one of the more surprising implications of the baseline model without homeownership:  The richer you are, the lower is the optimal share of your portfolio in risky assets.  This implication of the model does not match the available data well.  The conclusion is easy to reverse by introducing a bequest motive in which bequests are a luxury good; but how exactly such a motive should be constructed is by no means a settled question, either among financial planners or among academic researchers.  It is a topic we hope to address in future releases of our toolkit.

